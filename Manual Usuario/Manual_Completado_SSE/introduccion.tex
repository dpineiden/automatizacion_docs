\chapter{Presentación del Problema}

\section*{Introducción}

Este documento consiste en un \textit{manual de usuario} para utilizar la plantilla base que es el conjunto de información base para generar de forma automatizada los documentos FL33 y R08. Y además de una \textit{explicación extendida} de todo el proceso que ocurre en esa transformación.

La plantilla base consiste en seis secciones en que será necesario completar la información.

Esta consiste en datos relacionados a los parámetros que se toman en ciertas estaciones (o puntos de muestreo) en un proyecto. Estos parámetros se recuperan en envases que están codificados según corresponda y se envían a los laboratorios que tienen la disponibilidad para realizar los análisis.

Existe un orden de agrupación de la información dada a partir del laboratorio asignado a cada parámetro, con esto se hace posible extraer los datos y crear tablas ordenadas según esta estructura. Cada laboratorio en particular determina si cierta información es asignada para un \textbf{FL33} o un \textbf{R08} (orden de compra).

La forma en que se ordena cada tabla está determinada por la estructura de plantilla de cada documento final, de esta manera, si se logra ordenar el conjunto de datos de la planilla original de manera que siga un orden de agrupación asignado por el \textbf{FL33} será posible completar la información a través de una plantilla referencial, lo mismo para el documento \textbf{R08}.

\section*{Objetivo General}

Implementar un sistema computacional de generación automática de documentos FL33 y R08 diseñados por CEA, a partir de una planilla de cálculo que contiene la información de laboratorios, parámetros y estaciones entre otras.

\section*{Objetivos específicos}

\begin{itemize}
	\item{Estudiar estructura de datos de los documentos CEA.}
	\begin{itemize}
		\item \textbf{FL33}: Solicitud de Servicio Interno a Laboratorio.
		\item \textbf{R08}: Orden de Compra.
	\end{itemize}
	\item{Estudiar la estructura y composición de la información de planilla de cálculo base.}
	\item{Estudiar las herramientas a usar para implementar la solución}
	\begin{itemize}
		\item Scripts en Bash
		\item Scripts en Python
		\item Plantillas de documentos en formato Oasis Document Format (ODF)
	\end{itemize}
	\item{Implementar el proceso de extracción de datos}
	\item{Implementar el proceso de completado de plantillas}
\end{itemize}

\section*{Estructura del Documento}

La estructura de los capítulos siguientes obedece al orden planteado de acuerdo a los \textit{Objetivos específicos}.

El usuario que finalmente complete la información de la planilla base puede acceder directamente al capítulo II y III para completarla, no siendo necesario revisar el resto de los capítulos.

Quien se interese en continuar el desarrollo o modificar algún componente base en la estructura de datos y lograr la continuidad de está metodología automatizada de lograr los documentos debe estudiar los capítulos IV, V y VI y, necesariamente, sus \textit{referencias}. 